\chapter{Считывающие Макросы}\label{chapter_reading_macros}
\section{Время-Работы как Время-Чтения}\label{section_run-time_at_read-time}

\begin{quote}
Синтаксический сахар вызывает рак точек с запятой

- Алан Перлис
\end{quote}

Не один только Лисп даёт прямой доступ к коду, который может разбираться на структуры из cons ячеек, но, кроме этого Лисп предоставляет доступ к символам, составляющим ваши программы, ещё до того, как программы будут представлять из себя структуру. Обычный макрос работает с программой в форме дерева. Кроме обычного макроса существует специальный тип макросов, называемых \emph{считывающими макросами (read macro)}, оперирующими сырыми символами, из которых состоит ваша программа.

Если нам понадобиться определить не-Лисповский синтаксис в Лиспе, то в этом случае не имеет смысла использовать Лисповский считыватель - он предназначен только для чтения Лиспа. Считывающий макрос - это устройство, используемое для обработки не-Лисп синтаксиса, после считывающего макроса в дело вступает Лисп считыватель. Причина, по которой Лисп считыватель является более мощным чем считыватели в других языках заключается в том, что Лисп даёт нам возможность \emph{перехватывать (hooks)} каждый аспект поведения считывателя. В частности, Лисп позволяет вам \emph{расширять} считыватель, таким образом, что не-Лисп объекты будут прочитываться и преобразовываться в Лисп объекты. Также, как вы строите ваши приложения поверх Лиспа, расширяя его с помощью макросов и функций, также и Лисп приложения могут, а чаще всего и просачиваются в это измерение расширяемости. Когда это происходит, то становится возможным чтение любого символа на основе синтаксиса с помощью Лисп считывателя, а это означает что вы добавили этот синтаксис в Лисп.

В то время, когда трансформация кода, выполняемая обычными макросами используется для вставки Лисп кода в новый Лисп код, считывающие макросы могут быть написаны для вставки не-Лисп кода в Лисп код. Подобно обычным макросам, считывающие макросы реализуются с помощью функций, благодаря этому мы имеем доступ ко всей мощи Лисп окружения. Подобно макросам, увеличивающим продуктивность через создание более удобного предметно ориентированного языка, считывающие макросы увеличивают продуктивность позволяя сокращать выражения в точку где они больше не являются выражениями.

Если всё что нам нужно для того, чтобы разобрать эти не-Лисповские предметно-ориентированные языки - это написать короткий считывающий макрос, то может быть эти не-Лисп языки на самом деле являются умно замаскированным Лиспом. Если XML может прямо считываться Лисп считывателем [XML-AS-READ-MACRO], то может быть XML - это просто видоизменённый Лисп. Подобным образом считывающие макросы могут использоваться для чтения регулярных выражений и SQL запросов прямо в Лисп, то может быть эти языки на самом деле являются Лиспом. Это нечёткое различие между кодом и данными, Лиспом и не-Лиспом, являются источником многих интересных философских проблем, возникающих перед Лисп программистами с самого начала зарождения Лиспа.

Базовый считывающий макрос, встроенный в Common Lisp - это \emph{\#.} - вычисляющий макрос в момент чтения. Этот считывающий макрос позволяет вам встраивать объекты в читаемые, не сериализируемые формы, но создаваемые с помощью Лисп кода. Один забавный пример - это создание формы, содержащей изменяющиеся значения при каждом вызове.

\begin{verbatim}
CL-USER> '(football-game
           (game-started-at
            #.(get-internal-real-time))
           (coin-flip
            #.(if (zerop (random 2)) 'heads 'tails)))
(FOOTBALL-GAME (GAME-STARTED-AT 177030) (COIN-FLIP TAILS))
\end{verbatim}

Несмотря на то, что это то-же самое выражение, эта форма считывается по разному раз за разом:

\begin{verbatim}
CL-USER> '(football-game
           (game-started-at
            #.(get-internal-real-time))
           (coin-flip
            #.(if (zerop (random 2)) 'heads 'tails)))
(FOOTBALL-GAME (GAME-STARTED-AT 385713) (COIN-FLIP HEADS))
\end{verbatim}

Заметьте, что две формы окружённые \emph{\#.} вычисляются в момент чтения, а не тогда, когда вычисляется форма. Полный список был сформирован после того, как они были вычислены, а предыдущая и последующая эквивалентность (определённая \textbf{equal}'ом) может быть обнаружена через повторное вычисление последней прочитанной формы и сравнения её с предыдущими результатами, с помощью вспомогательных переменных \emph{*}, \emph{+} в REPL\footnote{Переменная \emph{*} содержит значение, являющееся результатом предыдущей формы, а переменная \emph{+} содержит значение последней формы.}:

\begin{verbatim}
CL-USER> (equal * (eval +))
T
\end{verbatim}

Следует обратить внимание на то, что эти формы вычисляются в момент чтения, этим они отличаются от использования обратной кавычки, более подробно мы рассмотрим обратную кавычку в следующем разделе. Мы можем вычислить подобную форму с помощью использования обратных кавычек:

\begin{verbatim}
CL-USER> `(football-game
           (game-started-at
            ,(get-internal-real-time))
           (coin-flip
            ,(if (zerop (random 2)) 'heads 'tails)))
(FOOTBALL-GAME (GAME-STARTED-AT 722283) (COIN-FLIP TAILS))
\end{verbatim}

но в этом случае будет происходить вычисление в различные результаты, поскольку обратная кавычка производит считывание в виде кода для вычисления:

\begin{verbatim}
CL-USER> (equal * (eval +))
NIL ;если вы не очень быстрый и удачливый
\end{verbatim}

\section{Обратная Кавычка}\label{section_backquote}

\emph{Обратная кавычка}, иногда называемая как \emph{квазикавычка (quasiquote)}\footnote{Scheme программисты называют её квазикавычкой а Common Lisp программисты - обратной кавычкой.} и отображаемая как \verb"`" - это относительно новый элемент в промышленных диалектах Лиспа. Эта концепция по-прежнему абсолютно чужда для не-Лисп языков программирования.

У обратной кавычки странная история развития. Обратная кавычка развивалась параллельно с Лиспом. Есть сведения [QUASIQUOTATION] о том, что раньше никто не верил в то, что вложенные обратные кавычки будут работать правильно, пока умные программисты не выяснили что обратные кавычки правильно выполняют свою работу - идеи людей о том, что было верным оказались неверными. Известно, что трудно понять работу вложенной обратной кавычки. Даже Стил (Steele), отец Common Lisp'а, жалуется на это [CLTL2—P530].

В принципе Лисп не нуждается в обратной кавычке. Всё, что можно выполнить с помощью обратной кавычки - можно выполнять с помощью других функций, предназначенных для постройки списков. Однако, обратная кавычка настолько полезна при программировании макросов, а в Лиспе под программированием понимается программирование макросов, что Лисп профессионалы очень активно ею пользуются.

Вначале мы должны разобраться с обычным закавычиванием. В Лиспе, мы используем префикс формы в виде символа кавычки (') для того, чтобы информировать Лисп о том, что следующая форма должна рассматриваться как сырые данные, а не как код, который нужно вычислить. Или, если выразиться иначе, результатом вычисления кавычки в коде будет возвращение формы. Иногда мы говорим что кавычка \emph{останавливает} или \emph{отключает} вычисление формы.

В Лиспе обратная кавычка может использоваться как замена кавычки. Исключая некоторые специальные символы, называемые символами \emph{раскавычивания (unquote)} и могущие появиться в форме, обратная кавычка останавливает вычисление тем-же способом, что и кавычка. Как следует из названия символы раскавычивания изменяют семантику вычисления. Иногда мы говорим что раскавычивание \emph{перезапускает} или \emph{включает} вычисление формы.

Есть три главных типа раскавычивания: обычное раскавычивание, объединяющее раскавычивание и деструктивное, объединяющее раскавычивание.

Для выполнения обычного раскавычивания, мы используем оператор запятую:

\begin{verbatim}
CL-USER> (let ((s 'hello))
           `(,s world))
(HELLO WORLD)
\end{verbatim}

Хотя выражение, которое мы раскавычили является простым вычислением символа, \textbf{s}, на месте этого символа может быть любое Лисп выражение, вычисляемое в нечто значимое для любого контекста, в котором оно появляется в шаблоне обратной кавычки. Каким бы ни был результат, он будет вставлен в результирующий список в car позиции того места в котором он появился в шаблоне обратной кавычки.

В нотации Лисповской формы мы можем использовать \textbf{.} если мы хотим явно вставить что-либо cdr'ное в создаваемую списковую структуру. Если здесь мы вставим список, то результирующей формой обратной кавычки будет корректный список. Но, если здесь мы вставим что-нибудь другое, то мы получим новую не-списковую структуру.

Такое-же поведение доступно нам везде, в том числе внутри обратной кавычки\footnote{Поскольку обратная кавычка использует (почти ту же) стандартную функцию \textbf{read} что и везде.}. Благодаря архитектуре обратной кавычки мы можем раскавычивать элементы даже в такой позиции:

\begin{verbatim}
CL-USER> (let ((s '(b c d)))
           `(a .,s))
(A B C D)
\end{verbatim}

Вставка списков в \emph{cdr} позиции создаваемого списка из шаблона обратной кавычки оказалось весьма универсальной операцией, поэтому был сделан следующий шаг названный как объединяющее раскавычивание. Вышеприведённая комбинация \textbf{.,} полезна, но не способна вставлять элементы в середину списка. Для этого у нас есть оператор объединяющее раскавычивание:

\begin{verbatim}
CL-USER> (let ((s '(b c d)))
           `(a ,@s e))
(A B C D E)
\end{verbatim}

Ни \textbf{.,} ни \textbf{,@} не модифицируют сращиваемый список. Для примера, после вычисления обратной кавычки в обеих предыдущих формах, \textbf{s} по прежнему будет привязан к трёх элементному списку \textbf{(B C D)}. Хотя это не является строго определяемым стандартом, в форме \textbf{(A B C D)}, \textbf{(B C D)} может являться общей структурой с объединённым списком \textbf{s}. Однако, структура списка \textbf{(A B C D E)} гарантирует, что она будет свежесозданной при вычислении обратной кавычки, поскольку \textbf{,@} запрещено модифицировать сращенные списки. Объединяющее раскавычивание - не деструктивная операция, поскольку в целом нам нужно думать об обратной кавычке как о компонуемом шаблоне для создания списков. Деструктивная модификация списковой структуры не свежесозданных данных при каждом вычислении обратно закавыченного кода может иметь нежелательные эффекты в плане будущих расширений.

Однако, в Common Lisp есть деструктивная версия сращиваемого раскавычивания, которое вы можете использовать везде, где можно использовать сращивающее раскавычивание. Для деструктивного раскавычивания используйте \textbf{,.} . Деструктивное сращивание работает также, как и обычное сращивание, за исключением того, что сращиваемый список может быть модифицирован в процессе вычисления шаблона обратной кавычки. И хотя отличие от обычного раскавычивания выражается всего лишь в одном символе, эта нотация более умно использует символ \textbf{.} из \emph{cdr} позиционного раскавычивания \textbf{.,} рассмотренного выше.

Для того, чтобы увидеть это в действии, в этом примере мы деструктивно модифицируем список указанный в \emph{to-splice}:

\begin{verbatim}
CL-USER> (defvar to-splice '(B C D))
TO-SPLICE
CL-USER> `(A ,.to-splice E)
(A B C D E)
CL-USER> to-splice
(B C D E)
\end{verbatim}

Выполнение де\-струк\-тив\-ной модификации сращиваемых списков может быть опасной операцией. Рассмотрим следующее применение деструктивного сращивания:

\begin{verbatim}
(defun dangerous-use-of-bq ()
  `(a ,. '(b c d) e))
\end{verbatim}

При первом вызове \textbf{dangerous-use-of-bq} возвращается ожидаемый ответ: \textbf{(A B C D E)}. Но, поскольку эта функция использует деструктивное сращивание и модификацию не свежесгенерированных списков - закавыченный список - то мы можем ожидать возникновения различных нежелательных последствий. В этом случае при втором вычислении \textbf{dangerous-use-of-bq} форма \textbf{(B C D)} будет представлять из себя форму \textbf{(B C D E)} и в момент, когда обратная кавычка попытается деструктивно срастить этот список с остатком шаблона обратной кавычки \textbf{(E)} - его собственный хвост - будет создан список, содержащий \emph{цикл (cycle)}. Более детально мы обсудим циклы в \emph{разделе Циклические Выражения}.

Однако, есть много случаев в которых деструктивное сращивание является чрезвычайно безопасным. Не позволяйте \textbf{dangerous-use-of-bq} напугать вас если вам нужна повышенная эффективность в ваших формах обратной кавычки. Есть множество операций, создающих свежие списковые структуры, которые вам так или иначе могут не понадобиться. Например, сращивание результатов \emph{mapcar} настолько распространено и безопасно, что вполне может претендовать на программную идиому:

\begin{verbatim}
(defun safer-use-of-bq ()
  `(a
    ,.(mapcar #'identity '(b c d))
    e))
\end{verbatim}

Но, есть деталь, которая мешает этому. Чаще всего обратная кавычка используется для создания макросов, часть программирования на Лиспе, где менее важна скорость и более важна ясность. Если думать о \emph{побочных эффектах} в ваших операциях сращивания, то при создании и интерпретации макросов вам придётся часто отвлекаться на них, а это не стоит свеч. Эта книга придерживается обычных сращиваний. Чаще всего обратная кавычка используется для конструирования макросов, но, это не единственное их использование. Обратная кавычка - это удобный предметно ориентированный язык для смешивания списков, и ещё более удобным он становится с возможностью деструктивного сращивания.

Как работает обратная кавычка? Обратная кавычка - это считывающий макрос. Формы обратной кавычки читаются как код, который при вычислении, становится желаемым списком. Возвращаясь к примеру из предыдущего раздела о вычислении во время выполнения, мы можем отключить \emph{красивую печать (pretty printing)}, закавычить значение формы обратной кавычки и вывести его на печать для того, чтобы увидеть как читаются формы с обратной кавычкой\footnote{Мы возвращаем \textbf{t} и поэтому мы не видим значения, возвращаемого от \textbf{print}. \textbf{(values)} также универсальны.}:

\begin{verbatim}
CL-USER> (let (*print-pretty*);привязка к nil
           (print
            '`(football-game
            (game-started-at
             ,(get-internal-real-time))
            (coin-flip
             ,(if (zerop (random 2))
                  'heads
                  'tails))))
           t)

(SB-IMPL::BACKQ-LIST 
 (QUOTE FOOTBALL-GAME) 
 (SB-IMPL::BACKQ-LIST 
  (QUOTE GAME-STARTED-AT) 
  (GET-INTERNAL-REAL-TIME)) 
 (SB-IMPL::BACKQ-LIST 
  (QUOTE COIN-FLIP) 
  (IF (ZEROP (RANDOM 2)) 
      (QUOTE HEADS) 
      (QUOTE TAILS)))) 
T
\end{verbatim}

В этой, \emph{некрасиво напечатанной} форме, функция {\selectlanguage{english}\textbf{LISP::BACKQ-LIST}} идентична \textbf{list}, за исключением поведения, связанного с красивой печатью. Заметьте, что операторы запятая исчезли. Common Lisp очень либерален и поэтому позволяет читать обратные кавычки, также, как и допускает операции с использованием общих структур.

Кроме того обратная кавычка предоставляет много интересных решений забавной \emph{не совсем проблемы} написания Лисп выражений, вычисляющихся в самих себя. Эти выражения часто называются \emph{куайнами} в честь Уилларда Куайна, который занимался их широким изучением и кто, по сути, ввёл термин квазицитирования - альтернативное название обратной кавычки [FOUNDATIONS-P31-FOOTNOTE3]. Вот забавный пример куайна, от Майка МакМахонома (Mike McMahon) {\selectlanguage{english} [QUASIQUOTATION]}:

\begin{verbatim}
CL-USER> (let ((let '`(let ((let ',let))
                        ,let)))
           `(let ((let ',let)) ,let))
(LET ((LET
       '`(LET ((LET ',LET))
           ,LET)))
  `(LET ((LET ',LET))
     ,LET))
\end{verbatim}

Для сохранения вашего \emph{ментального прохода по коду}:

\begin{verbatim}
CL-USER> (equal * +)
T
\end{verbatim}

Упражнение: Почему в выражении ниже обратная кавычка расширяется в обычную кавычку? Разве оно не закавычено?

\begin{verbatim}
CL-USER> `'q
'Q
\end{verbatim}

\section{Чтение Строк}\label{section_reading_strings}

В Лиспе строки разграничиваются символами двойной кавычки (\verb|"|). Хотя строки могут содержать любые символы из символьного набора в вашей Лисп реализации, но вы не можете непосредственно вставлять определённые символы в строку. Если вы хотите вставить символ \verb|"| или символ \verb"\", то вам придётся использовать префикс в виде символа \verb"\". Это называется \emph{экранированием} символов. Ниже показан пример с введением строки, содержащей символы \verb|"| и \verb"\":

\begin{verbatim}
 * "Contains \" and \\."
"Contains \" and \\."
\end{verbatim}

Принцип работы очевиден, но иногда печать символов \verb"\" становится утомительным и порождает ошибки. Это Лисп, и если нам что-то не нравится, то мы в силах внести наши изменения, это не просто легко сделать, но, ещё и приветствуется. Придерживаясь этой мысли, эта книга представляет считывающий макрос под названием \emph{\#"} или \emph{sharp-double-quote}. Этот считывающий макрос предназначен для создания строк, содержащих символы \verb|"| и \verb"\" без необходимости экранирования.

Листинг 4.1: SHARP-DOUBLE-QUOTE\label{listing_4.1}
\hrule
\begin{verbatim}
(defun |#"-reader| (stream sub-char numarg)
  (declare (ignore sub-char numarg))
  (let (chars)
    (do ((prev (read-char stream) curr)
         (curr (read-char stream) (read-char stream)))
        ((and (char= prev #\") (char= curr #\#)))
      (push prev chars))
    (coerce (nreverse chars) 'string)))

(set-dispatch-macro-character
 #\# #\" #'|#"-reader|)
\end{verbatim}
\hrule

\emph{Sharp-double-quote}\footnote{Наше соглашение об именовании нижележащих функций считывающих макросов с символами основывается на символах считывающего макроса, похожих на Стиловский считыватель \textbf{\#"-reader} в CLtL2.} начинает чтение строки непосредственно после вызова следующих символов: \emph{\#} и \emph{"}. Чтение будет продолжаться символ за символом до тех пор, пока в последовательности не встретятся два символа \emph{"} и \emph{\#}. Когда будет обнаружена завершающая последовательность, то будет возвращена строка, содержащая все символы между \emph{\#"} и \emph{"\#}. Считывающий макрос \emph{sharp-double-quote} создан для работы с битовыми строками, но Common Lisp позволяет нам использовать этот макрос передавая битовую строку в считывающий макрос \emph{\#*} [EARLY-CL-VOTES]. 

Вот пример использования нашего нового \emph{sharp-double-quote}:

\begin{verbatim}
 * #"Contains " and \."#

"Contains \" and \\."
\end{verbatim}

Учтите, что при печати строки REPL по прежнему будет использовать символ \verb|"| в качестве разграничителя, поэтому символы \verb|"| и \verb"\" по прежнему будут экранироваться в печатаемой строке. Эти строки по прежнему будут прочитываться так, как будто символы в них были экранированы вручную.

Но, иногда \emph{\#"} оказывается недостаточно хорошим. Например: в этом, только что прочитанном вами, параграфе U-Языка я вставил следующую последовательность символов: \emph{"\#}. По этой причине этот параграф не будет ограничиваться \emph{\#"} и \emph{"\#}. А поскольку я ненавижу экранируемые элементы, то поверьте мне на слово, здесь не будет разграничения обычными двойными кавычками.

Нам нужен макрос, который бы мог дать нам возможность модифицировать разграничитель для каждого используемого контекста. Как это часто бывает нам не нужно идти далеко за примером, достаточно взглянуть на язык Ларри Уолла - Perl чтобы почерпнуть вдохновения для архитектуры программных сокращений. Perl - это прекрасный, чудесно спроектированный язык, содержащий большое количество идей, которые могут быть \emph{украдены} Лиспом. В некотором роде, Лисп - это снежный ком, который катится по идеям из других языков программирования и вбирает их в себя, делая эти идеи своей частью\footnote{Наиболее цитируемым примером являются объекты, но кроме них существуют ещё бесчисленное множество других примеров, таких как функция \textbf{format} из FORTRAN'а.}.

Считывающий макрос \textbf{\#>} непосредственно вдохновлён оператором Perl'а \textbf{<<}. 

Листинг 4.2: SHARP-GREATER-THAN\label{listing_4.2}
\hrule
\begin{verbatim}
(defun |#>-reader| (stream sub-char numarg)
  (declare (ignore sub-char numarg))
  (let (chars)
    (do ((curr (read-char stream)
               (read-char stream)))
        ((char= #\newline curr))
      (push curr chars))
    (let* ((pattern (nreverse chars))
           (pointer pattern)
           (output))
      (do ((curr (read-char stream)
                 (read-char stream)))
          ((null pointer))
        (push curr output)
        (setf pointer
              (if (char= (car pointer) curr)
		  (cdr pointer)
                pattern))
        (if (null pointer)
	    (return)))
      (coerce
       (nreverse
	(nthcdr (length pattern) output))
       'string))))

(set-dispatch-macro-character
 #\# #\> #'|#>-reader|)
\end{verbatim}
\hrule

Этот оператор позволяет Perl программистам определять строку текста, которая будет служить разграничителем для цитируемой строки. \textbf{\#>} читает символы до тех пор, пока не найдёт символ перехода на новую строку, затем читает символы один-за-другим, до тех пор, пока не встретиться последовательность символов идентичная символам обнаруженным сразу после \textbf{\#>} и до новой строки.

Например:

\begin{verbatim}
 * #>END
I can put anything here: ", \, "#, and ># are
no problem. The only thing that will terminate
the reading of this string is...END

"I can put anything here: \", \\, \"#, and ># are
no problem. The only thing that will terminate
the reading of this string is..."
\end{verbatim}
