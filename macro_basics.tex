\chapter{Основы макросов}\label{chapter_macro_basics}
\section{Итеративная Разработка}\label{section_iterative-development}
\begin{quote}
Лисп помог целому ряду одарённых людей размышлять таким образом, какой ранее не был им доступен.

- Эдсгер Дейкстра
\end{quote}

Конструирование макросов - это итеративный процесс: все сложные макросы начинались из более простых макросов. Отталкиваясь от этой идеи, можно сказать что макросы создаются подобно скульптуре из куска камня. Если реализация макроса оказывается недостаточно гибкой, либо результатом оказывается недостаточно эффективное или опасное расширение, то в этом случае профессиональный программист макросов слегка модифицирует макрос, добавляя функционал или исправляя ошибки до тех пор, пока макрос не будет удовлетворять всем требованиям.

Необходимость итеративного процесса в конструировании макросов возникает отчасти потому, что итеративный процесс программирования - это наиболее универсальный и эффективный способ программирования, а другая причина заключается в том, что программирование макросов - это наиболее сложный вид программирования. Поскольку при программировании макросов программисту приходится думать сразу о нескольких уровнях кода, исполняемого в различные моменты времени, то вопросы сложности увеличиваются гораздо быстрее, чем в остальных типах программирования. Итеративный процесс позволяет убедиться в том, что ваша концептуальная модель наиболее полно соответствует тому, что планируется создать. Если бы мы создавали макросы без такой обратной связи, то конструирование макросов стало бы значительно более трудным делом.

В этой главе мы напишем несколько базовых макросов и ознакомимся с двумя основными концепциями макросов: \emph{предметно - ориентированные языки (domain specific languages)} и \emph{структуры управления (control structures)}. После того, как мы изучим эти основные понятия о макросах, мы вернёмся назад и обсудим процесс создания макросов с помощью макросов. На протяжении всей книги мы будем использовать такие техники как захват переменной и введение свободной переменной, кроме этого мы ознакомимся с новым, более удобным синтаксисом определения Лисп макросов.

