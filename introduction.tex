\chapter{Введение}\label{chapter_introduction}
\section{Макросы}\label{section_macros}
\begin{quote}
Ядро Lisp'а занимает в некотором роде локальный оптимум в пространстве языков программирования.

— Скромные слова Джона МакКарти, создателя Lisp'а
\end{quote}
Эта книга о программировании \emph{макросов} на Lisp-е. Большинство книг о программировании дают лишь беглый обзор материала. Эта книга построена в виде руководств и примеров, написанных таким образом, чтобы вы могли максимально быстро и эффективно программировать сложные макросы. Освоение макросов - это финальный шаг, отделяющий посредственного Lisp-программиста от Lisp-профессионала.

Макросы - это единственное самое большое преимущество Lisp-а и единственная важнейшая деталь, которая может присутствовать в других языках программирования. С помощью макросов вы можете делать то, что попросту невозможно в других языках. Поскольку макросы могут использоваться для преобразования Lisp-а в другие языки программирования и обратно, то программисты, изучившие макросы, могут обнаружить, что все остальные языки являются просто оболочкой поверх Lisp-а. Это \emph{далеко идущий} вывод. Lisp примечателен тем, что программирование на нём - это программирование на высочайшем уровне. В то время, когда большинство языков изобретают и применяют синтаксические и семантические правила, Lisp универсален и пластичен. С Lisp-ом вы создаёте правила.

История Lisp'а более богата и обширна, нежели история других языков программирования. Над разработкой этого языка работали одни из лучших компьютерных учёных. Благодаря их трудам возник самый мощный и универсальный язык программирования. Lisp содержит много стандартов, несколько замечательных реализаций с открытым исходным кодом и макросы, с которыми не сравнятся другие языки программирования. В этой книге используется только COMMON LISP[ANSI—CL][CLTL2], но большое количество идей можно легко портировать на такие Lisp'ы, как Scheme[R5RS]. Тем не менее, ознакомившись с этой книгой, вы увидите, что для написания макросов стоит использовать COMMON LISP. В то время, когда остальные Lisp'ы хороши для других целей, COMMON LISP - это выбор профессионала, работающего с макросами.

Архитекторы COMMON LISP-а проделали замечательную работу при проектировании языка. Учитывая качество реализации COMMON LISP, на данный момент, - это лучшая среда для разработки программ с удивительно малым количеством оговорок. Как программист вы всегда можете рассчитывать на COMMON LISP и будьте уверены что в этом языке всё сделано так, как надо. Несмотря на то, что архитекторы и авторы реализаций выполнили свою работу на отлично, есть пробелы в объяснении того, почему язык реализован таким, а не другим способом. Для многих программистов COMMON LISP - это огромное коллекция непонятных особенностей, поэтому эти программисты переходят к более привычным, для них, языкам, так и не познав истинной мощи макросов. Эта книга может быть использована в качестве путеводителя по многим замечательным особенностям удивительного языка программирования - COMMON LISP. Большинство языков спроектированы так, чтобы их можно было легко реализовать; COMMON LISP спроектирован для создания мощных программ. Я искренне надеюсь, что создатели COMMON LISP оценят эту книгу как наиболее завершённый и доступный источник сведений об особенностях макросов и что эта книга будет ощутимой каплей в океан тем о макросах.

История макросов начинается почти с историей Lisp-а. Макросы изобретены Тимоти Хартом [MACRO-DEFINITIONS] в 1963 году. Тем не менее, большинство Lisp программистов не используют всю мощь макросов, а остальные программисты вообще не используют макросы. Это всегда остаётся загадкой для продвинутых лисперов. Если макросы настолько хороши, то почему их не используют все программисты? Самые умные и наиболее решительные программисты в конечном счёте приходят к макросам Лиспа. Для того, чтобы понять возможности макросов, необходимо понять что есть в Лиспе и чего нет в других языках. А это в свою очередь требует знания других, менее мощных языков. К сожалению многие программисты теряют желание учиться после того, как они освоят несколько других языков и таким образом никогда не узнают что такое макросы и их возможности. Но, несмотря на это, некоторый процент программистов задумывается над возможностью создания программ, пишущих другие программы - то есть, они приходят к макросам. А поскольку Лисп - это лучший язык для создания макросов, самые умные, самые решительные и самые любознательные программисты всегда приходят к Лиспу.

Программисты из числа верхнего процентиля всегда будут малым числом, несмотря на то, что общая популяция программистов растёт. Люди из мира программирования видят несколько примеров мощи макросов, а число людей понимающих макросы ещё меньше, но всё постепенно меняется. Поскольку макросы увеличивают производительность в разы, \emph{век макросов} наступает, независимо от того, готов к ним мир или нет. Цель этой книги - стать первой линией подготовки к неизбежному будущему: миру макросов. Будьте готовы.

Вокруг макросов распространено следующее мнение - использовать их только при необходимости. Причина этому - некоторые макросы сложно понимать, в макросы могут закрасться очень трудно определяемые ошибки и в случае, когда вы обо всём думаете как о функциях, то макросы могут вас неприятно удивить. Это не дефекты Лисповской системы макросов, а общие черты программирования макросов. Как и в случае с любой технологией: чем более мощен инструмент - тем больше способов его неправильного использования. А среди всех программных конструкций макросы Лиспа - это наиболее мощный инструмент.

Можно провести параллель между изучением макросов в Лиспе и изучением указателей в языке программирования C. Большинство начинающих программистов C легко усваивают большую часть языка. Функции, типы, переменные, арифметические выражения: все они имеют параллели с предыдущим интеллектуальным опытом начинающих программистов, начиная с математики уровня начальной школы и заканчивая экспериментами с простейшими языками программирования. Но, большинство новичков в C упираются в кирпичную стену при изучении указателей.

Указатели являются второй натурой опытного C программиста. Многие опытные C программисты считают что полное понимание указателей необходимо для правильного использования C. Поскольку указатели являются фундаментальной идеей, многие опытные C программисты не рекомендуют ограничивать использование указателей при обучении или для стилистической красоты. Несмотря на это многие новички в C считают указатели ненужным усложнением и избегают их использования, так возникает \emph{симптом FORTRAN-а (пренебрежение полезными особенностями ''вне зависимости от языка'')}. Болезнью является игнорирование особенностей языка, а не плохой программистский стиль. Если же особенности языка освоены в полном объёме, то корректный стиль программирования становится очевидным. Не нужно стремиться выработать какой-либо стиль программирования, это касается всех языков, - вспомогательная тема этой книги. Стиль нужен только в том случае, когда отсутствует понимание\footnote{Следствие этого - эффективное использование чего-либо только одним, подсмотренным где-либо, способом при отсутствии должного понимания природы вещей.}.

Подобно указателям C, макросы - это особенность Лиспа, которая часто остаётся плохо понятой, и поэтому распространены не совсем верные представления о макросах. Если при работе с макросами вы полагаетесь на такие высказывания как:

\begin{quote}
Макросы изменяют синтаксис Лисп кода.

Макросы работают в дереве разбора ваших программ.

Используйте макросы только тогда, когда с задачей не справляются функции.
\end{quote}
то скорее всего вы упустили из виду общую картину, что даёт программирование макросов. Именно это и исправляет эта книга.

Хорошие справочники и руководства по макросам можно пересчитать по пальцам. Одним из хороших книг по макросам является книга Пола Грэма \emph{On Lisp} [On-Lisp]. Рекомендуется к прочтению от корки до корки всем, кто интересуется макросами. \emph{On Lisp} и остальные труды Грэма послужили толчком к написанию книги, которую вы сейчас читаете. Благодаря Полу Грэму и остальным людям, писавшим о Лиспе, мощь макросов широко обсуждается, но, к сожалению, всё же остаётся также широко не понятой. Несмотря на то, что просто прочитав книгу \emph{On Lisp} можно узнать много интересного о макросах, некоторые программисты связывают проблемы программирования с макросами. В то время, когда \emph{On Lisp} показывает вам различные виды макросов, эта книга расскажет вам как использовать эти макросы.

Написание макросов - это итеративный процесс, связанный с размышлениями. Все сложные макросы начинаются из простых макросов, прошедших через долгую серию улучшений и тестирований. Кроме того, знание о том, где применить макросы приходит с накоплением опыта написания макросов. При написании программ человек следует некоторой системе и процессу. Каждый программист представляет себе некую концептуальную модель работы инструментов программирования и создаёт код исходя из результатов вытекающих из этой модели. Программист, обладающий интеллектом начнёт задумываться о программировании, как о логической процедуре, и придёт к мысли, о процессе автоматизации программирования. После этого программист будет готовиться к процессам автоматизации.

Важным шагом к пониманию макросов является следующее: если писать код без тщательного планирования и без приложения значительных усилий, то в результате код будет нашпигован множественными шаблонами и негибкими абстракциями. Такое положение вы можете увидеть в любых больших программах. Дублируемые участки кода и слишком усложнённый код - это отсутствие правильных абстракций у авторов кода. Эффективное использование макросов подразумевает под собой признание проблемы в виде повторяющихся шаблонов и абстракций, после этого следует создание \emph{кода, помогающего вам писать код}. Но этого не достаточно для того, чтобы понять как писать макросы; профессиональный Лисп программист хочет знать зачем писать макросы.

C программисты, только пришедшие в Лисп, часто делают ошибку считая что главное предназначение макросов в улучшении эффективности кода в момент выполнения\footnote{C программисты делают эту ошибку по причине того, что "макро система" действительно хороша для этих целей, но главное назначение макросов не в этом.}. Да, довольно часто макросы используются именно для этой задачи, но наиболее общей целью использования макросов является упрощение процесса программирования. В большинстве программах шаблоны просто избыточно копируются, а абстракции используются в не достаточной мере, правильно спроектированные макросы могут вывести выразительность программирования на ещё больший уровень. Там, где остальные языки оказываются ограниченными и конкретными, Лисп остаётся универсальным и гибким.

Эта книга не введение в Лисп. Темы и материал подобраны для профессиональных программистов в не - Лисп языках, интересующихся пользой, которую можно извлечь из макросов и для студентов, посредственно знающих Лисп, которые готовы по настоящему изучить самую сильную сторону Лиспа. Подразумевается, что вы посредственно знаете Лисп, и от вас не требуется глубокого знания замыканий и макросов.

Эта книга не только о теории. Все примеры, приведённые здесь, полностью функциональны, пригодны для использования и могут помочь вам улучшить ваше программирование здесь и сейчас. Эта книга о применении передовых программистских технологий для улучшения вашего программирования. Во многих книгах посвящённых программированию сознательно применяется простой стиль программирования в угоду доступности. В этой книге материал преподаётся с полным применением всего языка. Кроме этого приведённые примеры кода используют эзотерические особенности COMMON LISP, большинство из которых будут описываться при использовании. Если вы прочитали и поняли\footnote{Конечно, вы можете не соглашаться с этим утверждением.} всё, что описано в \emph{главе 2, Замыкания, на странице \pageref{chapter_closures}} и в \emph{главе 3, Основы Макросов, на странице \pageref{chapter_macro_basics}}, то можете считать что вы прошли среднюю стадию понимания Лиспа.

Одной из частей Лиспа является самостоятельные открытия, эта книга не лишит вас удовольствия от экспериментов. Имейте в виду, что материал в этой книге подаётся очень быстро, много быстрее, чем вы сможете усвоить. Для того, чтобы понять некоторые участки кода, приведённые здесь, вам придётся обращаться к другим справочникам и руководствам по COMMON LISP'у. После изучения базовых понятий мы перейдём прямо к последним исследованиям, посвящённым макросам, большинство из которых граничат с большой неизведанной областью. Эта книга фокусируется на \emph{комбинировании макросов}. Эта тема имеет пугающую репутацию, а хорошо понять эту тему способны не все программисты. Комбинирование макросов включает в себя наиболее обширные и плодородные исследования в языках программирования. Исследователями было выжато почти всё из типов, объектов и пролого-подобной логики, но программирование макросов остаётся огромной, зияющей чёрной дырой. Никто не знает до конца что лежит по ту сторону. Всё, что мы знаем на данный момент - это то, что макросы сложны, пугающи и проявляют неограниченный потенциал. В отличие от многих программистских идей макросы не являются ни академической концепцией для производства бесполезных теоретических публикаций, ни пустым модным словом относящемуся к программному обеспечению компаний. Макросы - это лучшие друзья хакеров. Макросы делают ваши программы умнее, а не труднее. Большинство программистов, начавших изучение макросов приходят к выводу что программирование без макросов - это не программирование.

Многие книги посвящённые Лиспу написаны в пропагандистском ключе, но я совершенно равнодушно отношусь к публичным призывам использовать Лисп. Лисп не собирается уходить. Я был бы счастлив если бы мог использовать Лисп в качестве \emph{секретного оружия} на протяжении всей своей карьеры программиста. Эта книга имеет только одну цель - вдохновить на изучение и исследование, также, как я был вдохновлён \emph{On Lisp}'ом. Я надеюсь, что смогу вдохновить читателей этой книги и смогу насладиться ещё более лучшими Лисповскими макро инструментами и ещё более интересными книгами о макросах Лиспа.

Да пребудет с вами сила Лиспа,

ваш покорный автор,

Даг Хойт.
\section{U-Язык}\label{section_u_language}
\section{Утилиты Лисп}\label{section_the_lisp_utility}
\section{Лицензия}\label{section_license}
\section{Благодарности}\label{section_thanks}
